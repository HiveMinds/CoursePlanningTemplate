\section{}\label{sec:q6}
